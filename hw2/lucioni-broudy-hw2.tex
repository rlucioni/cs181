\documentclass[solution, letterpaper]{cs121}

\usepackage{tikz-qtree}
\usepackage{graphicx}

%% Please fill in your name and collaboration statement here.
%\newcommand{\studentName}{Renzo Lucioni and Daniel Broudy}
%\newcommand{\collaborationStatement}{I collaborated with...}
\newcommand{\solncolor}{red}
\begin{document}

\header{2}{March 1, 2013, at 12:00 PM}{}{}

%%%%%%%%%%%%%%%%%%%%%%%%%%%%%%%%%%%%%%%%%%%%%%%%%%%%
\problem{} For this problem, we assume that an image is a $3 \times 3$ array of pixels (inputs) arranged as follows:
\begin{center}
\begin{tabular}{ l c r }
  $x_1$ & $x_2$ & $x_3$ \\
  $x_4$ & $x_5$ & $x_6$ \\
  $x_7$ & $x_8$ & $x_9$ \\
\end{tabular}
\end{center}
Each $x_k$ in the input vector {\textbf{\emph{x}}} represents a pixel state, and can be either 1 (on) or 0 (off). Let the weight vector {\textbf{\emph{w}}} $= (w_0, w_1, \ldots, w_9)$ represent the weights of the corresponding inputs in the array. Let $x_0$ be the special instance that is always fixed at 1, and let $w_0$ be the threshold of the perceptron activation function.

\subproblem Bright-or-dark: at least 75\% of the pixels are on, or at least 75\% of the pixels are off. \\

Assume for the purpose of contradiction that there exists some perceptron recognizing bright-or-dark with weights {\textbf{\emph{w}}}. Say we turn off pixels $x_1$ and $x_2$ and leave the rest on. Greater than 75\% of the pixels are on, and so the perceptron outputs 1, implying that $s > 0$, where $s$ is the weighted sum of the inputs. That is,
\[ s = w_0 + w_3 + \ldots + w_9 > 0 \]
Say we also turn off pixel $x_3$, leaving the rest on. Now there are not 75\% of pixels on or off, and so the perceptron outputs -1, implying that $s \leq 0$. That is,
\[ s = w_0 + w_4 + \ldots + w_9 \leq 0 \]
The removal of $w_3$ causes $s$ to become less than or equal to 0, hence it must be that $w_3 > 0$. \\

Now we consider what happens when we turn on pixels $x_1$ and $x_2$ and leave the rest off. Greater than 75\% of the pixels are off, and so the perceptron outputs 1, implying that $s > 0$. That is,
\[ s = w_0 + w_1 + w_2 > 0 \]
Say we also turn on pixel $x_3$, leaving the rest off. Now there are not 75\% of pixels on or off, and so the perceptron outputs -1, implying that $s \leq 0$. That is,
\[ s = w_0 + w_1 + w_2 + w_3 \leq 0 \]
The addition of $w_3$ causes $s$ to become less than or equal to 0, hence it must be that $w_3 < 0$. But this is a contradiction, and therefore there exists no perceptron recognizing bright-or-dark.

\subproblem Top-bright: a larger fraction of pixels is on in the top row than in the bottom two rows. \\

A perceptron that recognizes bright-or-dark takes a vector of 9 inputs {\textbf{\emph{x}}} in which an on pixel is represented with a 1 and an off pixel is represented with a 0. The first top row will be the first three of the nine and the bottom two rows will be the last six. The perceptron is defined by the set of weights {\textbf{\emph{w}}} in which the first three weights are  $\frac{1}{3}$, the last six weights are $\frac{-1}{6}$, and the threshold activation function is 
\begin{equation*}
  g(s)=\begin{cases}
    +1, & \text{if $s  >  0$}.\\
    -1, & \text{otherwise}.
  \end{cases}
\end{equation*}

This equation works by calculating the top fraction (by adding $\frac{1}{3}$ for every 1) and then subtracting the bottom fraction (by subtracting  $\frac{1}{6}$ for every 1). 

\subproblem Connected: the set of pixels that are on is connected. \\




%%%%%%%%%%%%%%%%%%%%%%%%%%%%%%%%%%%%%%%%%%%%%%%%%%%%
\problem{}

%%%%%%%%%%%%%%%%%%%%%%%%%%%%%%%%%%%%%%%%%%%%%%%%%%%%
\problem{}

%%%%%%%%%%%%%%%%%%%%%%%%%%%%%%%%%%%%%%%%%%%%%%%%%%%%
\problem{}



\end{document}



